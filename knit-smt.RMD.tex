% Options for packages loaded elsewhere
\PassOptionsToPackage{unicode}{hyperref}
\PassOptionsToPackage{hyphens}{url}
%
\documentclass[
]{article}
\usepackage{amsmath,amssymb}
\usepackage{lmodern}
\usepackage{ifxetex,ifluatex}
\ifnum 0\ifxetex 1\fi\ifluatex 1\fi=0 % if pdftex
  \usepackage[T1]{fontenc}
  \usepackage[utf8]{inputenc}
  \usepackage{textcomp} % provide euro and other symbols
\else % if luatex or xetex
  \usepackage{unicode-math}
  \defaultfontfeatures{Scale=MatchLowercase}
  \defaultfontfeatures[\rmfamily]{Ligatures=TeX,Scale=1}
\fi
% Use upquote if available, for straight quotes in verbatim environments
\IfFileExists{upquote.sty}{\usepackage{upquote}}{}
\IfFileExists{microtype.sty}{% use microtype if available
  \usepackage[]{microtype}
  \UseMicrotypeSet[protrusion]{basicmath} % disable protrusion for tt fonts
}{}
\makeatletter
\@ifundefined{KOMAClassName}{% if non-KOMA class
  \IfFileExists{parskip.sty}{%
    \usepackage{parskip}
  }{% else
    \setlength{\parindent}{0pt}
    \setlength{\parskip}{6pt plus 2pt minus 1pt}}
}{% if KOMA class
  \KOMAoptions{parskip=half}}
\makeatother
\usepackage{xcolor}
\IfFileExists{xurl.sty}{\usepackage{xurl}}{} % add URL line breaks if available
\IfFileExists{bookmark.sty}{\usepackage{bookmark}}{\usepackage{hyperref}}
\hypersetup{
  pdftitle={BMI Distribution},
  pdfauthor={20210522 - Antoinette Bonifacia Duweeja de Lima.},
  hidelinks,
  pdfcreator={LaTeX via pandoc}}
\urlstyle{same} % disable monospaced font for URLs
\usepackage[margin=1in]{geometry}
\usepackage{color}
\usepackage{fancyvrb}
\newcommand{\VerbBar}{|}
\newcommand{\VERB}{\Verb[commandchars=\\\{\}]}
\DefineVerbatimEnvironment{Highlighting}{Verbatim}{commandchars=\\\{\}}
% Add ',fontsize=\small' for more characters per line
\usepackage{framed}
\definecolor{shadecolor}{RGB}{248,248,248}
\newenvironment{Shaded}{\begin{snugshade}}{\end{snugshade}}
\newcommand{\AlertTok}[1]{\textcolor[rgb]{0.94,0.16,0.16}{#1}}
\newcommand{\AnnotationTok}[1]{\textcolor[rgb]{0.56,0.35,0.01}{\textbf{\textit{#1}}}}
\newcommand{\AttributeTok}[1]{\textcolor[rgb]{0.77,0.63,0.00}{#1}}
\newcommand{\BaseNTok}[1]{\textcolor[rgb]{0.00,0.00,0.81}{#1}}
\newcommand{\BuiltInTok}[1]{#1}
\newcommand{\CharTok}[1]{\textcolor[rgb]{0.31,0.60,0.02}{#1}}
\newcommand{\CommentTok}[1]{\textcolor[rgb]{0.56,0.35,0.01}{\textit{#1}}}
\newcommand{\CommentVarTok}[1]{\textcolor[rgb]{0.56,0.35,0.01}{\textbf{\textit{#1}}}}
\newcommand{\ConstantTok}[1]{\textcolor[rgb]{0.00,0.00,0.00}{#1}}
\newcommand{\ControlFlowTok}[1]{\textcolor[rgb]{0.13,0.29,0.53}{\textbf{#1}}}
\newcommand{\DataTypeTok}[1]{\textcolor[rgb]{0.13,0.29,0.53}{#1}}
\newcommand{\DecValTok}[1]{\textcolor[rgb]{0.00,0.00,0.81}{#1}}
\newcommand{\DocumentationTok}[1]{\textcolor[rgb]{0.56,0.35,0.01}{\textbf{\textit{#1}}}}
\newcommand{\ErrorTok}[1]{\textcolor[rgb]{0.64,0.00,0.00}{\textbf{#1}}}
\newcommand{\ExtensionTok}[1]{#1}
\newcommand{\FloatTok}[1]{\textcolor[rgb]{0.00,0.00,0.81}{#1}}
\newcommand{\FunctionTok}[1]{\textcolor[rgb]{0.00,0.00,0.00}{#1}}
\newcommand{\ImportTok}[1]{#1}
\newcommand{\InformationTok}[1]{\textcolor[rgb]{0.56,0.35,0.01}{\textbf{\textit{#1}}}}
\newcommand{\KeywordTok}[1]{\textcolor[rgb]{0.13,0.29,0.53}{\textbf{#1}}}
\newcommand{\NormalTok}[1]{#1}
\newcommand{\OperatorTok}[1]{\textcolor[rgb]{0.81,0.36,0.00}{\textbf{#1}}}
\newcommand{\OtherTok}[1]{\textcolor[rgb]{0.56,0.35,0.01}{#1}}
\newcommand{\PreprocessorTok}[1]{\textcolor[rgb]{0.56,0.35,0.01}{\textit{#1}}}
\newcommand{\RegionMarkerTok}[1]{#1}
\newcommand{\SpecialCharTok}[1]{\textcolor[rgb]{0.00,0.00,0.00}{#1}}
\newcommand{\SpecialStringTok}[1]{\textcolor[rgb]{0.31,0.60,0.02}{#1}}
\newcommand{\StringTok}[1]{\textcolor[rgb]{0.31,0.60,0.02}{#1}}
\newcommand{\VariableTok}[1]{\textcolor[rgb]{0.00,0.00,0.00}{#1}}
\newcommand{\VerbatimStringTok}[1]{\textcolor[rgb]{0.31,0.60,0.02}{#1}}
\newcommand{\WarningTok}[1]{\textcolor[rgb]{0.56,0.35,0.01}{\textbf{\textit{#1}}}}
\usepackage{graphicx}
\makeatletter
\def\maxwidth{\ifdim\Gin@nat@width>\linewidth\linewidth\else\Gin@nat@width\fi}
\def\maxheight{\ifdim\Gin@nat@height>\textheight\textheight\else\Gin@nat@height\fi}
\makeatother
% Scale images if necessary, so that they will not overflow the page
% margins by default, and it is still possible to overwrite the defaults
% using explicit options in \includegraphics[width, height, ...]{}
\setkeys{Gin}{width=\maxwidth,height=\maxheight,keepaspectratio}
% Set default figure placement to htbp
\makeatletter
\def\fps@figure{htbp}
\makeatother
\setlength{\emergencystretch}{3em} % prevent overfull lines
\providecommand{\tightlist}{%
  \setlength{\itemsep}{0pt}\setlength{\parskip}{0pt}}
\setcounter{secnumdepth}{-\maxdimen} % remove section numbering
\ifluatex
  \usepackage{selnolig}  % disable illegal ligatures
\fi

\title{BMI Distribution}
\author{20210522 - Antoinette Bonifacia Duweeja de Lima.}
\date{4/15/2023}

\begin{document}
\maketitle

\hypertarget{r-markdown}{%
\subsection{R Markdown}\label{r-markdown}}

This is an R Markdown document. Markdown is a simple formatting syntax
for authoring HTML, PDF, and MS Word documents. For more details on
using R Markdown see \url{http://rmarkdown.rstudio.com}.

When you click the \textbf{Knit} button a document will be generated
that includes both content as well as the output of any embedded R code
chunks within the document. You can embed an R code chunk like this:
ated the plot.

\begin{Shaded}
\begin{Highlighting}[]
\CommentTok{\#Question 1)}
\CommentTok{\#a)}
\CommentTok{\#set seed for reproducibility.}
\FunctionTok{set.seed}\NormalTok{(}\DecValTok{123}\NormalTok{)}
\CommentTok{\#generating 10000 weight values from N(60, 32).}
\NormalTok{w }\OtherTok{\textless{}{-}} \FunctionTok{rnorm}\NormalTok{(}\DecValTok{10000}\NormalTok{, }\AttributeTok{mean =} \DecValTok{60}\NormalTok{, }\AttributeTok{sd =} \DecValTok{4}\NormalTok{)}
\CommentTok{\#generate 10000 height values from N(1.6, 0.12).}
\NormalTok{h }\OtherTok{\textless{}{-}} \FunctionTok{rnorm}\NormalTok{(}\DecValTok{10000}\NormalTok{, }\AttributeTok{mean =} \FloatTok{1.6}\NormalTok{, }\AttributeTok{sd =} \FloatTok{0.3464}\NormalTok{)}
\CommentTok{\#calculating the BMI values.}
\NormalTok{bmi }\OtherTok{\textless{}{-}}\NormalTok{ w }\SpecialCharTok{/}\NormalTok{ h}\SpecialCharTok{\^{}}\DecValTok{2}

\CommentTok{\#b)}
\CommentTok{\#plotting the histogram of BMI values.}
\FunctionTok{hist}\NormalTok{(bmi, }\AttributeTok{breaks =} \DecValTok{50}\NormalTok{, }\AttributeTok{col =} \StringTok{"red"}\NormalTok{, }\AttributeTok{xlab =} \StringTok{"BMI"}\NormalTok{, }\AttributeTok{main =} \StringTok{"BMI PLOT"}\NormalTok{)}
\end{Highlighting}
\end{Shaded}

\includegraphics{knit-smt_files/figure-latex/unnamed-chunk-1-1.pdf}

\begin{Shaded}
\begin{Highlighting}[]
\CommentTok{\#c)}
\CommentTok{\#calculating the mean and the variance of BMI.}
\FunctionTok{mean}\NormalTok{(bmi)}
\end{Highlighting}
\end{Shaded}

\begin{verbatim}
## [1] 28.00382
\end{verbatim}

\begin{Shaded}
\begin{Highlighting}[]
\FunctionTok{var}\NormalTok{(bmi)}
\end{Highlighting}
\end{Shaded}

\begin{verbatim}
## [1] 331.5486
\end{verbatim}

\begin{Shaded}
\begin{Highlighting}[]
\CommentTok{\#d)}
\CommentTok{\#estimating P(BMI \textgreater{}= 25).}
\FunctionTok{mean}\NormalTok{(bmi }\SpecialCharTok{\textgreater{}=} \DecValTok{25}\NormalTok{)}
\end{Highlighting}
\end{Shaded}

\begin{verbatim}
## [1] 0.4471
\end{verbatim}

\begin{Shaded}
\begin{Highlighting}[]
\CommentTok{\#Question2)}
\CommentTok{\#a)}
\CommentTok{\#Set seed for reproducibility.}
\FunctionTok{set.seed}\NormalTok{(}\DecValTok{123}\NormalTok{)}

\CommentTok{\#Defining probabilities for winning a point under each service rule.}
\NormalTok{p\_win\_A }\OtherTok{\textless{}{-}} \FloatTok{0.55}
\NormalTok{p\_win\_B }\OtherTok{\textless{}{-}} \FloatTok{0.40}

\CommentTok{\#Function to simulate a game.}
\NormalTok{simulate\_game }\OtherTok{\textless{}{-}} \ControlFlowTok{function}\NormalTok{(p\_win\_serve\_A, p\_win\_serve\_B) \{}
\NormalTok{  score\_player1 }\OtherTok{\textless{}{-}} \DecValTok{0}
\NormalTok{  score\_player2 }\OtherTok{\textless{}{-}} \DecValTok{0}
  
  \CommentTok{\#Simulating game until one player wins 2 points.}
  \ControlFlowTok{while}\NormalTok{ (score\_player1 }\SpecialCharTok{\textless{}} \DecValTok{2} \SpecialCharTok{\&}\NormalTok{ score\_player2 }\SpecialCharTok{\textless{}} \DecValTok{2}\NormalTok{) \{}
    \ControlFlowTok{if}\NormalTok{ (}\FunctionTok{sample}\NormalTok{(}\FunctionTok{c}\NormalTok{(}\ConstantTok{TRUE}\NormalTok{, }\ConstantTok{FALSE}\NormalTok{), }\DecValTok{1}\NormalTok{, }\AttributeTok{prob =} \FunctionTok{c}\NormalTok{(p\_win\_serve\_A, }\DecValTok{1} \SpecialCharTok{{-}}\NormalTok{ p\_win\_serve\_A))) \{}
\NormalTok{      score\_player1 }\OtherTok{\textless{}{-}}\NormalTok{ score\_player1 }\SpecialCharTok{+} \DecValTok{1}
\NormalTok{    \} }\ControlFlowTok{else}\NormalTok{ \{}
\NormalTok{      score\_player2 }\OtherTok{\textless{}{-}}\NormalTok{ score\_player2 }\SpecialCharTok{+} \DecValTok{1}
\NormalTok{    \}}
\NormalTok{  \}}
  
  \FunctionTok{return}\NormalTok{(score\_player1 }\SpecialCharTok{\textgreater{}}\NormalTok{ score\_player2)}
\NormalTok{\}}

\CommentTok{\#Simulating 1000 games under service rule A.}
\NormalTok{n }\OtherTok{\textless{}{-}} \DecValTok{1000}
\NormalTok{winners\_A }\OtherTok{\textless{}{-}} \FunctionTok{replicate}\NormalTok{(n, }\FunctionTok{simulate\_game}\NormalTok{(}\AttributeTok{p\_win\_serve\_A =}\NormalTok{ p\_win\_A, }\AttributeTok{p\_win\_serve\_B =}\NormalTok{ p\_win\_A))}

\CommentTok{\#Simulating 1000 games under service rule B.}
\NormalTok{winners\_B }\OtherTok{\textless{}{-}} \FunctionTok{replicate}\NormalTok{(n, }\FunctionTok{simulate\_game}\NormalTok{(}\AttributeTok{p\_win\_serve\_A =}\NormalTok{ p\_win\_B, }\AttributeTok{p\_win\_serve\_B =}\NormalTok{ p\_win\_B))}

\CommentTok{\#Calculating the winning probabilities under each service rule.}
\NormalTok{winning\_prob\_A }\OtherTok{\textless{}{-}} \FunctionTok{mean}\NormalTok{(winners\_A)}
\NormalTok{winning\_prob\_B }\OtherTok{\textless{}{-}} \FunctionTok{mean}\NormalTok{(winners\_B)}

\CommentTok{\#Printing the results.}
\FunctionTok{cat}\NormalTok{(}\StringTok{"a) Winning probability of Player 1 under service rule A:"}\NormalTok{, winning\_prob\_A, }\StringTok{"}\SpecialCharTok{\textbackslash{}n}\StringTok{"}\NormalTok{)}
\end{Highlighting}
\end{Shaded}

\begin{verbatim}
## a) Winning probability of Player 1 under service rule A: 0.588
\end{verbatim}

\begin{Shaded}
\begin{Highlighting}[]
\FunctionTok{cat}\NormalTok{(}\StringTok{"a) Winning probability of Player 1 under service rule B:"}\NormalTok{, winning\_prob\_B, }\StringTok{"}\SpecialCharTok{\textbackslash{}n}\StringTok{"}\NormalTok{)}
\end{Highlighting}
\end{Shaded}

\begin{verbatim}
## a) Winning probability of Player 1 under service rule B: 0.355
\end{verbatim}

\begin{Shaded}
\begin{Highlighting}[]
\CommentTok{\#b)}
\CommentTok{\#Function to simulate a game and return the number of points played.}
\NormalTok{simulate\_game\_length }\OtherTok{\textless{}{-}} \ControlFlowTok{function}\NormalTok{(p\_win\_serve\_A, p\_win\_serve\_B) \{}
\NormalTok{  score\_player1 }\OtherTok{\textless{}{-}} \DecValTok{0}
\NormalTok{  score\_player2 }\OtherTok{\textless{}{-}} \DecValTok{0}
\NormalTok{  num\_points }\OtherTok{\textless{}{-}} \DecValTok{0}
  
  \CommentTok{\#Simulating game until one player wins 2 points.}
  \ControlFlowTok{while}\NormalTok{ (score\_player1 }\SpecialCharTok{\textless{}} \DecValTok{2} \SpecialCharTok{\&}\NormalTok{ score\_player2 }\SpecialCharTok{\textless{}} \DecValTok{2}\NormalTok{) \{}
\NormalTok{    num\_points }\OtherTok{\textless{}{-}}\NormalTok{ num\_points }\SpecialCharTok{+} \DecValTok{1}
    \ControlFlowTok{if}\NormalTok{ (}\FunctionTok{sample}\NormalTok{(}\FunctionTok{c}\NormalTok{(}\ConstantTok{TRUE}\NormalTok{, }\ConstantTok{FALSE}\NormalTok{), }\DecValTok{1}\NormalTok{, }\AttributeTok{prob =} \FunctionTok{c}\NormalTok{(p\_win\_serve\_A, }\DecValTok{1} \SpecialCharTok{{-}}\NormalTok{ p\_win\_serve\_A))) \{}
\NormalTok{      score\_player1 }\OtherTok{\textless{}{-}}\NormalTok{ score\_player1 }\SpecialCharTok{+} \DecValTok{1}
\NormalTok{    \} }\ControlFlowTok{else}\NormalTok{ \{}
\NormalTok{      score\_player2 }\OtherTok{\textless{}{-}}\NormalTok{ score\_player2 }\SpecialCharTok{+} \DecValTok{1}
\NormalTok{    \}}
\NormalTok{  \}}
  
  \FunctionTok{return}\NormalTok{(num\_points)}
\NormalTok{\}}

\CommentTok{\#Simulating 1000 games and calculate the expected length of a game under service rule A.}
\CommentTok{\#Number of games to simulate.}
\NormalTok{n }\OtherTok{\textless{}{-}} \DecValTok{1000} 
\CommentTok{\#Probability of winning a point under service rule A.}
\NormalTok{p\_win\_A }\OtherTok{\textless{}{-}} \FloatTok{0.55} 

\NormalTok{game\_lengths\_A }\OtherTok{\textless{}{-}} \FunctionTok{replicate}\NormalTok{(n, }\FunctionTok{simulate\_game\_length}\NormalTok{(}\AttributeTok{p\_win\_serve\_A =}\NormalTok{ p\_win\_A, }\AttributeTok{p\_win\_serve\_B =}\NormalTok{ p\_win\_A))}
\NormalTok{expected\_length\_A }\OtherTok{\textless{}{-}} \FunctionTok{mean}\NormalTok{(game\_lengths\_A)}

\CommentTok{\#Simulating 1000 games and calculate the expected length of a game under service rule B.}
\CommentTok{\#Probability of winning a point under service rule B.}
\NormalTok{p\_win\_B }\OtherTok{\textless{}{-}} \FloatTok{0.40} 

\NormalTok{game\_lengths\_B }\OtherTok{\textless{}{-}} \FunctionTok{replicate}\NormalTok{(n, }\FunctionTok{simulate\_game\_length}\NormalTok{(}\AttributeTok{p\_win\_serve\_A =}\NormalTok{ p\_win\_B, }\AttributeTok{p\_win\_serve\_B =}\NormalTok{ p\_win\_B))}
\NormalTok{expected\_length\_B }\OtherTok{\textless{}{-}} \FunctionTok{mean}\NormalTok{(game\_lengths\_B)}

\CommentTok{\#Printing the results.}
\FunctionTok{cat}\NormalTok{(}\StringTok{"b) Expected length of a game under service rule A:"}\NormalTok{, expected\_length\_A, }\StringTok{"points}\SpecialCharTok{\textbackslash{}n}\StringTok{"}\NormalTok{)}
\end{Highlighting}
\end{Shaded}

\begin{verbatim}
## b) Expected length of a game under service rule A: 2.511 points
\end{verbatim}

\begin{Shaded}
\begin{Highlighting}[]
\FunctionTok{cat}\NormalTok{(}\StringTok{"b) Expected length of a game under service rule B:"}\NormalTok{, expected\_length\_B, }\StringTok{"points}\SpecialCharTok{\textbackslash{}n}\StringTok{"}\NormalTok{)}
\end{Highlighting}
\end{Shaded}

\begin{verbatim}
## b) Expected length of a game under service rule B: 2.514 points
\end{verbatim}

\begin{Shaded}
\begin{Highlighting}[]
\CommentTok{\#c)}
\CommentTok{\#Based on the simulation results, the following is the estimated length of a game for each service rule:}
\CommentTok{\# {-} The expected length of a game under service rule A is projected to be expected\_length\_A points (where the server is the winner of the preceding point). This means that when service rule A is followed, it takes roughly expected\_length\_A points for one player to win 2 points and the game.}
\CommentTok{\# {-} The expected length of a game under service rule B is calculated to be expected\_length\_B points (where the server is the loser of the preceding point). This suggests that when service rule B is followed, it takes about expected\_length\_B points for one player to win 2 points and win the game.}

\CommentTok{\#d)}
\CommentTok{\#1)Player performance consistency: The simulation assumes that the players\textquotesingle{} performance is consistent throughout the game and that the probability of earning a point under each service rule appropriately represent their actual performance. If player performance varies over time or due to external circumstances such as fatigue or injury, the simulation findings may be invalid.}
\CommentTok{\#2)Players with equal skill levels: The simulation assumes that both players have equal skill levels, with the only difference being the service rule being followed. If the participants have major skill discrepancies, the results may not correctly represent the game\textquotesingle{}s outcomes. In actuality, player skill levels might fluctuate, influencing game dynamics and outcomes.}
\CommentTok{\#3)Point independence: The simulation assumes that the outcome of each point is independent of prior points, which means that the chances of winning a point do not alter based on the outcome of previous points. This may not always be the case in real{-}world settings, as players may be swayed by momentum, psychological variables, or strategy alterations based on recent outcomes.}
\end{Highlighting}
\end{Shaded}

\begin{Shaded}
\begin{Highlighting}[]
\CommentTok{\#Question3)}
\CommentTok{\#a)}
\CommentTok{\#Setting the parameters.}
\CommentTok{\#number of steps.}
\NormalTok{n }\OtherTok{\textless{}{-}} \DecValTok{1000} 
\CommentTok{\#probability of +1 step.}
\NormalTok{p }\OtherTok{\textless{}{-}} \FloatTok{0.5} 
\CommentTok{\#probability of {-}1 step.}
\NormalTok{q }\OtherTok{\textless{}{-}} \FloatTok{0.5} 

\CommentTok{\#Initializing the random walk.}
\CommentTok{\#vector to store the random walk.}
\NormalTok{S }\OtherTok{\textless{}{-}} \FunctionTok{rep}\NormalTok{(}\DecValTok{0}\NormalTok{, n}\SpecialCharTok{+}\DecValTok{1}\NormalTok{) }
\CommentTok{\#set seed for reproducibility.}
\FunctionTok{set.seed}\NormalTok{(}\DecValTok{123}\NormalTok{) }

\CommentTok{\#Simulating the random walk.}
\ControlFlowTok{for}\NormalTok{ (i }\ControlFlowTok{in} \DecValTok{1}\SpecialCharTok{:}\NormalTok{n) \{}
  \CommentTok{\#Generating a random step (+1 or {-}1).}
\NormalTok{  step }\OtherTok{\textless{}{-}} \FunctionTok{sample}\NormalTok{(}\FunctionTok{c}\NormalTok{(}\DecValTok{1}\NormalTok{, }\SpecialCharTok{{-}}\DecValTok{1}\NormalTok{), }\AttributeTok{size =} \DecValTok{1}\NormalTok{, }\AttributeTok{prob =} \FunctionTok{c}\NormalTok{(p, q))}
  
  \CommentTok{\#Updating the random walk.}
\NormalTok{  S[i}\SpecialCharTok{+}\DecValTok{1}\NormalTok{] }\OtherTok{\textless{}{-}}\NormalTok{ S[i] }\SpecialCharTok{+}\NormalTok{ step}
\NormalTok{\}}

\CommentTok{\#Ploting the random walk.}
\FunctionTok{plot}\NormalTok{(}\DecValTok{1}\SpecialCharTok{:}\NormalTok{(n}\SpecialCharTok{+}\DecValTok{1}\NormalTok{), S, }\AttributeTok{type =} \StringTok{"l"}\NormalTok{, }\AttributeTok{xlab =} \StringTok{"n"}\NormalTok{, }\AttributeTok{ylab =} \StringTok{"Sn"}\NormalTok{, }\AttributeTok{main =} \StringTok{"Random Walk"}\NormalTok{,}\AttributeTok{col =} \StringTok{"blue"}\NormalTok{)}
\end{Highlighting}
\end{Shaded}

\includegraphics{knit-smt_files/figure-latex/unnamed-chunk-3-1.pdf}

\begin{Shaded}
\begin{Highlighting}[]
\CommentTok{\#b)}
\CommentTok{\#As n goes to infinity, the random variable Sn will display the following properties:}

\CommentTok{\#Random walk behavior: The random walk Sn, n 1 may display particular tendencies as n goes to infinity, such as becoming more symmetric and centered around 0, with the absolute values of the steps becoming less relevant relative to the overall behavior of the random walk. Depending on the precise probabilities of the steps and their relationship to one another, the random walk may also display qualities such as recurrence or transience.}

\CommentTok{\#Distribution: Due to the central limit theorem, when n goes to infinity, the distribution of Sn may converge to a Gaussian distribution (also known as a normal distribution).The central limit theorem says that the sum of a large number of independent and identically distributed random variables tends to be regularly distributed, regardless of the form of the initial distribution.}

\CommentTok{\#Mean: Sn\textquotesingle{}s mean can be computed by multiplying the expected value of each individual random variable by n, given that the expected value exists. Because Xn has a mean of 0 (due to the equal odds of +1 and {-}1 steps), the mean of Sn will be 0 for all values of n.}

\CommentTok{\#Variance: Sn\textquotesingle{}s variance can be computed by adding the variances of each individual random variable and multiplying by n, given that the variances exist. In this situation, because Xn has variance 1 (owing to equal probabilities of +1 and {-}1 steps), Sn has variance n for all values of n.}
\end{Highlighting}
\end{Shaded}

\begin{Shaded}
\begin{Highlighting}[]
\CommentTok{\#Question4)}
\CommentTok{\#a)}
\CommentTok{\#Initializing transition probability matrix.}
\NormalTok{P }\OtherTok{\textless{}{-}} \FunctionTok{matrix}\NormalTok{(}\FunctionTok{c}\NormalTok{(}\FloatTok{0.1}\NormalTok{, }\FloatTok{0.2}\NormalTok{, }\FloatTok{0.7}\NormalTok{, }\FloatTok{0.2}\NormalTok{, }\FloatTok{0.4}\NormalTok{, }\FloatTok{0.4}\NormalTok{, }\FloatTok{0.1}\NormalTok{, }\FloatTok{0.3}\NormalTok{, }\FloatTok{0.6}\NormalTok{), }\AttributeTok{nrow =} \DecValTok{3}\NormalTok{)}

\CommentTok{\#Initialize state space and current state.}
\NormalTok{states }\OtherTok{\textless{}{-}} \FunctionTok{c}\NormalTok{(}\StringTok{"A"}\NormalTok{, }\StringTok{"B"}\NormalTok{, }\StringTok{"C"}\NormalTok{)}
\CommentTok{\#1 represents state A.}
\NormalTok{current\_state }\OtherTok{\textless{}{-}} \DecValTok{1}  

\CommentTok{\#Simulate 5 possible beer purchases for 10 weeks.}
\ControlFlowTok{for}\NormalTok{ (i }\ControlFlowTok{in} \DecValTok{1}\SpecialCharTok{:}\DecValTok{5}\NormalTok{) \{}
  \FunctionTok{cat}\NormalTok{(}\FunctionTok{paste0}\NormalTok{(}\StringTok{"Simulation "}\NormalTok{, i, }\StringTok{":}\SpecialCharTok{\textbackslash{}n}\StringTok{"}\NormalTok{))}
  \ControlFlowTok{for}\NormalTok{ (j }\ControlFlowTok{in} \DecValTok{1}\SpecialCharTok{:}\DecValTok{10}\NormalTok{) \{}
    \FunctionTok{cat}\NormalTok{(}\FunctionTok{paste0}\NormalTok{(}\StringTok{"Week "}\NormalTok{, j, }\StringTok{": "}\NormalTok{, states[current\_state], }\StringTok{"}\SpecialCharTok{\textbackslash{}n}\StringTok{"}\NormalTok{))}
\NormalTok{    current\_state }\OtherTok{\textless{}{-}} \FunctionTok{sample}\NormalTok{(}\DecValTok{1}\SpecialCharTok{:}\DecValTok{3}\NormalTok{, }\AttributeTok{size =} \DecValTok{1}\NormalTok{, }\AttributeTok{prob =}\NormalTok{ P[current\_state, ])}
\NormalTok{  \}}
  \FunctionTok{cat}\NormalTok{(}\StringTok{"}\SpecialCharTok{\textbackslash{}n}\StringTok{"}\NormalTok{)}
\NormalTok{\}}
\end{Highlighting}
\end{Shaded}

\begin{verbatim}
## Simulation 1:
## Week 1: A
## Week 2: B
## Week 3: C
## Week 4: A
## Week 5: A
## Week 6: A
## Week 7: B
## Week 8: C
## Week 9: A
## Week 10: B
## 
## Simulation 2:
## Week 1: B
## Week 2: C
## Week 3: A
## Week 4: B
## Week 5: C
## Week 6: A
## Week 7: A
## Week 8: C
## Week 9: C
## Week 10: C
## 
## Simulation 3:
## Week 1: A
## Week 2: B
## Week 3: A
## Week 4: C
## Week 5: A
## Week 6: A
## Week 7: B
## Week 8: C
## Week 9: C
## Week 10: A
## 
## Simulation 4:
## Week 1: C
## Week 2: A
## Week 3: C
## Week 4: C
## Week 5: A
## Week 6: C
## Week 7: C
## Week 8: A
## Week 9: C
## Week 10: C
## 
## Simulation 5:
## Week 1: C
## Week 2: B
## Week 3: C
## Week 4: C
## Week 5: C
## Week 6: A
## Week 7: B
## Week 8: B
## Week 9: B
## Week 10: A
\end{verbatim}

\begin{Shaded}
\begin{Highlighting}[]
\CommentTok{\#b)}
\CommentTok{\#Initializing transition probability matrix.}
\NormalTok{P }\OtherTok{\textless{}{-}} \FunctionTok{matrix}\NormalTok{(}\FunctionTok{c}\NormalTok{(}\FloatTok{0.1}\NormalTok{, }\FloatTok{0.2}\NormalTok{, }\FloatTok{0.7}\NormalTok{, }\FloatTok{0.2}\NormalTok{, }\FloatTok{0.4}\NormalTok{, }\FloatTok{0.4}\NormalTok{, }\FloatTok{0.1}\NormalTok{, }\FloatTok{0.3}\NormalTok{, }\FloatTok{0.6}\NormalTok{), }\AttributeTok{nrow =} \DecValTok{3}\NormalTok{)}

\CommentTok{\#Identifying starting state as A.}
\CommentTok{\#1 represents state A.}
\NormalTok{start\_state }\OtherTok{\textless{}{-}} \DecValTok{1}  

\CommentTok{\#Calculating probability of transitioning from A to A in four steps.}
\NormalTok{prob\_A\_to\_A\_4steps }\OtherTok{\textless{}{-}} \FunctionTok{as.numeric}\NormalTok{(}\FunctionTok{t}\NormalTok{(P[start\_state, ])) }\SpecialCharTok{\%\%}\NormalTok{ P }\SpecialCharTok{\%\%}\NormalTok{ P }\SpecialCharTok{\%*\%}\NormalTok{ P[start\_state, ]}

\CommentTok{\#Multiplying initial probability of being in state A with probability of transitioning from A to A in four steps.}
\NormalTok{prob\_A\_A\_5weeks }\OtherTok{\textless{}{-}} \DecValTok{1} \SpecialCharTok{*}\NormalTok{ prob\_A\_to\_A\_4steps}

\CommentTok{\#Print the results.}
\FunctionTok{cat}\NormalTok{(prob\_A\_A\_5weeks)}
\end{Highlighting}
\end{Shaded}

\begin{verbatim}
## 0.02 0.06 0.04
\end{verbatim}

\begin{Shaded}
\begin{Highlighting}[]
\CommentTok{\#Question5)}
\CommentTok{\#a)The state space of this HMM would be \{hot, cold\} representing the two possible weather states of this Model because those 2 are the hidden states in this HMM these are the 2 states of the weather and the emission which is not hidden will be E=\{Sandals,Flipflops,Boots\}}

\CommentTok{\#b)}
\CommentTok{\#Initializing transition probability matrix P.}
\NormalTok{P }\OtherTok{\textless{}{-}} \FunctionTok{matrix}\NormalTok{(}\FunctionTok{c}\NormalTok{(}\FloatTok{0.25}\NormalTok{, }\FloatTok{0.75}\NormalTok{, }\FloatTok{0.25}\NormalTok{, }\FloatTok{0.67}\NormalTok{, }\FloatTok{0.33}\NormalTok{, }\FloatTok{0.33}\NormalTok{), }\AttributeTok{nrow =} \DecValTok{2}\NormalTok{, }\AttributeTok{byrow =} \ConstantTok{TRUE}\NormalTok{)}

\CommentTok{\#Printing the transition probability matrix P.}
\FunctionTok{print}\NormalTok{(P)}
\end{Highlighting}
\end{Shaded}

\begin{verbatim}
##      [,1] [,2] [,3]
## [1,] 0.25 0.75 0.25
## [2,] 0.67 0.33 0.33
\end{verbatim}

\begin{Shaded}
\begin{Highlighting}[]
\CommentTok{\#c) }
\CommentTok{\#Initializing transition probability matrix P.}
\NormalTok{P }\OtherTok{\textless{}{-}} \FunctionTok{matrix}\NormalTok{(}\FunctionTok{c}\NormalTok{(}\FloatTok{0.25}\NormalTok{, }\FloatTok{0.75}\NormalTok{, }\FloatTok{0.25}\NormalTok{, }\FloatTok{0.67}\NormalTok{, }\FloatTok{0.33}\NormalTok{, }\FloatTok{0.33}\NormalTok{), }\AttributeTok{nrow =} \DecValTok{2}\NormalTok{, }\AttributeTok{byrow =} \ConstantTok{TRUE}\NormalTok{)}

\CommentTok{\#Probability of wearing sandals today (state 2, choice 1).}
\NormalTok{P\_sandals\_today }\OtherTok{\textless{}{-}}\NormalTok{ P[}\DecValTok{2}\NormalTok{, }\DecValTok{1}\NormalTok{]}

\CommentTok{\#Probability of transitioning from state 2 to state 2 (cold to cold).}
\NormalTok{P\_cold\_to\_cold }\OtherTok{\textless{}{-}}\NormalTok{ P[}\DecValTok{2}\NormalTok{, }\DecValTok{2}\NormalTok{]}

\CommentTok{\#Probability of wearing sandals tomorrow (state 2, choice 1).}
\NormalTok{P\_sandals\_tomorrow }\OtherTok{\textless{}{-}}\NormalTok{ P\_sandals\_today }\SpecialCharTok{*}\NormalTok{ P\_cold\_to\_cold}

\CommentTok{\#Printing the result.}
\FunctionTok{cat}\NormalTok{(}\StringTok{"Probability of wearing sandals both today and tomorrow:"}\NormalTok{, P\_sandals\_tomorrow)}
\end{Highlighting}
\end{Shaded}

\begin{verbatim}
## Probability of wearing sandals both today and tomorrow: 0.2211
\end{verbatim}

\end{document}
